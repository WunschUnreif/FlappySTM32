\renewenvironment{longtable}{\rowcolors{2}{LightGray}{white}\oldlongtable} {\endoldlongtable}
\chapter{开发环境搭建与工程模板}
\section{开发环境概述}
编写STM32程序的过程大致可以分为3个阶段:编写代码、编译链接、下载程序。其中,编写代码是使用C语言,结合标准外设库,编制源代码的阶段,编译链接是利用编译器,将源代码编译成STM32平台的可执行文件,而下载程序是利用ST提供的接口驱动,将电脑上的可执行文件传输至STM32的Flash程序存储器中的过程。
下面针对这3个阶段,分别介绍本文档推荐的工具。
\paragraph{编写代码} 
本文档推荐的代码编辑工具是微软公司推出的开源跨平台编辑器VS Code。这款编辑器拥有丰富的插件,配合这些插件可以实现对C语言很好的支持,代码补全的能力相当高,而且可以一边编写代码一边查错。此外,利用它的集成终端功能,可以在编辑器内执行命令行命令,完成编译和下载。
\paragraph{编译链接}
本文档推荐的编译器是arm-none-eabi-gcc, 是由GCC提供的专为ARM平台编译程序的交叉编译工具链,历史悠久,性能可靠,最重要的一点是这个编译器完全免费而且开源,使用它不必担心遭到起诉,是彻头彻尾的正版软件。相比之下,Keil、IAR等软件虽然也是STM32程序编写和编译的流行软件,但它们正版软件的价格高昂,而且打击盗版的力度较大,因此不推荐使用。
\paragraph{下载程序}
本文档推荐的下载程序的接口和软件为ST-Link,这是ST公司为STM32设计的下载与调试接口,只需要4根线就可以连接。此外,使用J-Link也可以完成程序下载的功能,但J-Link接口需要20线,且下载器价格较高,因此并不推荐使用。目前,ST-Link也是ST公司官方推荐的连接方式。

\section{软件的安装}
VS Code的安装方法较为简单,只需在官网上下载安装,之后安装几个与C/C++有关的插件即可。
\par 
arm-none-eabi-gcc可以在https://github.com/gnu-mcu-eclipse/arm-none-eabi-gcc中找到下载链接,可以下载对应操作系统编译好的版本,也可以下载源码后自行编译。如果下载的是编译好的版本,那么在解压后得到的目录下的bin文件夹中就会看到arm-none-eabi-gcc这个可执行文件。方便起见,可以将这个目录添加到系统的环境变量PATH中。对于macOS用户(部分Linux系统用户也适用),可以在终端执行
\par 
\begin{lstlisting}[language=bash, style=customStyleBashLight]
nano ~/.bash_profile
\end{lstlisting}
\par 
然后在文件末尾添加:
\par 
\begin{lstlisting}[language=bash, style=customStyleBashLight]
export PATH="/path/to/gcc/bin:$PATH"
\end{lstlisting}
\par 
保存退出后,再执行命令:
\par 
\begin{lstlisting}[language=bash, style=customStyleBashLight]
source ~/.bash_profile
\end{lstlisting}
\par
即可完成环境变量的改变。
\par 
st-link软件可以直接利用系统的包管理器安装,对于使用homebrew的macOS用户,只需在终端中执行
\par 
\begin{lstlisting}[language=bash, style=customStyleBashLight]
brew install st-link
\end{lstlisting}
\par
稍等片刻就完成了st-link软件的安装。

\paragraph{Windows系统下相关软件安装概述}
对于Windows系统,代码编辑器VS code和交叉编译器arm-none-eabi-gcc都有相应的版本,只需下载安装即可,而对于下载软件st-link,也可用Windows系统下功能相同的软件替代。当然,我们推荐使用正版软件。

\section{建立工程模版}
首先,假定我们已经下载了STM32F1系列的标准外设库,并且位于STDPeriphLib/目录下。接下来,我们需要新建一个目录,作为工程模板的根目录,不妨设这个目录为C8Template/。在C8Template/下,我们需要新建下列文件夹和文件:
\begin{center}
	\begin{longtable}[l]{| p{30mm} | p{30mm} | p{80mm} |}
		\caption{C8Template/下的内容}\\
		\hline 
		\rowcolor{Gray}
		\textbf{项目名称} & \textbf{类型} & \textbf{简要描述} \\
		\hline
		\endfirsthead
		
		\hline 
		\rowcolor{Gray}
		\textbf{项目名称} & \textbf{类型} & \textbf{简要描述} \\
		\hline
		\endhead
		
		CMSIS/ &  目录 & 与Cortex-M3内核有关的库文件所在目录 \\ 
		Startup/ & 目录 & 单片机启动代码所在目录 \\
		Lib/ & 目录 & 外设库所在目录 \\
		User/ & 目录 & 开发者编写的代码所在目录 \\
		makefile & 文本文件 & 构建工程的脚本 \\
		stm32\_flash.ld & 文本文件 & gcc需要的连接器脚本 \\
		\hline
	\end{longtable}
\end{center}
\par 
下面逐一介绍各个目录和文件的内容。
\subsection{CMSIS/}
这个目录下放置的是与单片机核心有关的库文件,需要从标准外设库的\\STDPeriphLib/Libraries/CMSIS/CM3/中拷贝以下5个文件:
\begin{itemize}
	\item CoreSupport/core\_cm3.c
	\item CoreSupport/core\_cm3.h
	\item DeviceSupport/ST/STM32F10x/system\_stm32f10x.c
	\item DeviceSupport/ST/STM32F10x/system\_stm32f10x.h
	\item DeviceSupport/ST/STM32F10x/stm32f10x.h
\end{itemize}
\subsection{Startup/}
这个目录下放置的是单片机的启动代码,需要从标准外设库的
STDPeriphLib/\\Libraries/CMSIS/CM3/DeviceSupport/ST/STM32F10x/startup/gcc\_ride7/
目录下拷贝所有的文件。
\subsection{Lib/}
这个目录下放置的是所有外设的库文件,需要拷贝标准外设库的
STDPeriphLib/\\Libraries/STM32F10x\_StdPeriph\_Driver/下的inc/和src/两个目录,此外,还需要创建一个名为makefile的文本文件,其内容如下:
\begin{lstlisting}[language=bash, style=customStyleMakefile, caption=Lib/makefile]
CC = arm-none-eabi-gcc
AR = arm-none-eabi-ar 
vpath %.c src ../CMSIS
vpath %.h inc
CFLAGS  = -g -O2 -Wall
CFLAGS += -mlittle-endian -mthumb -mcpu=cortex-m3 -mthumb-interwork
CFLAGS += -ffreestanding -nostdlib
CFLAGS += -Iinc -I../CMSIS -I../User
CFLAGS += -DUSE_STDPERIPH_DRIVER
CFLAGS += -DSTM32F10X_MD

SRCS = misc.c stm32f10x_adc.c stm32f10x_bkp.c stm32f10x_can.c \
			stm32f10x_cec.c stm32f10x_crc.c stm32f10x_dac.c \
			stm32f10x_dbgmcu.c stm32f10x_dma.c stm32f10x_exti.c \
			stm32f10x_flash.c stm32f10x_fsmc.c stm32f10x_gpio.c \
			stm32f10x_i2c.c stm32f10x_iwdg.c stm32f10x_pwr.c \
			stm32f10x_rcc.c stm32f10x_rtc.c stm32f10x_sdio.c \
			stm32f10x_spi.c stm32f10x_tim.c stm32f10x_usart.c \
			stm32f10x_wwdg.c
SRCS += core_cm3.c system_stm32f10x.c

OBJS = $(SRCS:.c=.o)

all: libstm32periLib.a

%.o : %.c
$(CC) $(CFLAGS) -c -o $@ $^

libstm32periLib.a : $(OBJS)
$(AR) -r $@ $(OBJS)

.PHONY: clean 
clean: 
rm -f $(OBJS)
\end{lstlisting}
虽然你可能不想一个字一个字的敲这一段脚本,但是这也是作者纯手写的,所以。。。

\subsection{/User}
这个目录下放置的是开发者可以改动、添加的代码文件,一般来说,开发STM32程序需要建立和编辑的代码都在这个目录下放置。
\par 
初始状态下,我们需要从标准外设库中拷贝以下4个文件:
\begin{itemize}
	\item STDPeriphLib/Project/STM32F10x\_StdPeriph\_Template/main.c
	\item STDPeriphLib/Project/STM32F10x\_StdPeriph\_Template/stm32f10x\_conf.h
	\item STDPeriphLib/Project/STM32F10x\_StdPeriph\_Template/stm32f10x\_it.c
	\item STDPeriphLib/Project/STM32F10x\_StdPeriph\_Template/stm32f10x\_it.h
\end{itemize}

\subsection{stm32\_flash.ld}
这个文件是链接器脚本,其模板如下:
\begin{lstlisting}[language=bash, style=customStyleMakefile, caption=stm32\_flash.ld]
/* Entry Point */
ENTRY(Reset_Handler)

/* Highest address of the user mode stack */
_estack = 0x20005000;    /* end of RAM */

_Min_Heap_Size = 0;      /* required amount of heap  */
_Min_Stack_Size = 0x400; /* required amount of stack */

/* Memories definition */
MEMORY
{
  RAM (xrw)		: ORIGIN = 0x20000000, LENGTH = 20K
  ROM (rx)		: ORIGIN = 0x8000000, LENGTH = 64K
}

/* Sections */
SECTIONS
{
  /* The startup code into ROM memory */
  .isr_vector :
  {
    . = ALIGN(4);
    KEEP(*(.isr_vector)) /* Startup code */
    . = ALIGN(4);
  } >ROM

  /* The program code and other data into ROM memory */
  .text :
  {
    . = ALIGN(4);
    *(.text)           /* .text sections (code) */
    *(.text*)          /* .text* sections (code) */
    *(.glue_7)         /* glue arm to thumb code */
    *(.glue_7t)        /* glue thumb to arm code */
    *(.eh_frame)

    KEEP (*(.init))
    KEEP (*(.fini))

    . = ALIGN(4);
    _etext = .;        /* define a global symbols at end of code */
    _exit = .;
  } >ROM

  /* Constant data into ROM memory*/
  .rodata :
  {
    . = ALIGN(4);
    *(.rodata)         /* .rodata sections (constants, strings, etc.) */
    *(.rodata*)        /* .rodata* sections (constants, strings, etc.) */
    . = ALIGN(4);
  } >ROM

  .ARM.extab   : { 
  	. = ALIGN(4);
  	*(.ARM.extab* .gnu.linkonce.armextab.*)
  	. = ALIGN(4);
  } >ROM
  
  .ARM : {
    . = ALIGN(4);
    __exidx_start = .;
    *(.ARM.exidx*)
    __exidx_end = .;
    . = ALIGN(4);
  } >ROM

  .preinit_array     :
  {
    . = ALIGN(4);
    PROVIDE_HIDDEN (__preinit_array_start = .);
    KEEP (*(.preinit_array*))
    PROVIDE_HIDDEN (__preinit_array_end = .);
    . = ALIGN(4);
  } >ROM
  
  .init_array :
  {
    . = ALIGN(4);
    PROVIDE_HIDDEN (__init_array_start = .);
    KEEP (*(SORT(.init_array.*)))
    KEEP (*(.init_array*))
    PROVIDE_HIDDEN (__init_array_end = .);
    . = ALIGN(4);
  } >ROM
  
  .fini_array :
  {
    . = ALIGN(4);
    PROVIDE_HIDDEN (__fini_array_start = .);
    KEEP (*(SORT(.fini_array.*)))
    KEEP (*(.fini_array*))
    PROVIDE_HIDDEN (__fini_array_end = .);
    . = ALIGN(4);
  } >ROM

  /* Used by the startup to initialize data */
  _sidata = LOADADDR(.data);

  /* Initialized data sections into RAM memory */
  .data : 
  {
    . = ALIGN(4);
    _sdata = .;        /* create a global symbol at data start */
    *(.data)           /* .data sections */
    *(.data*)          /* .data* sections */

    . = ALIGN(4);
    _edata = .;        /* define a global symbol at data end */
  } >RAM AT> ROM

  /* Uninitialized data section into RAM memory */
  . = ALIGN(4);
  .bss :
  {
    /* This is used by the startup in order to initialize the .bss secion */
    _sbss = .;         /* define a global symbol at bss start */
    __bss_start__ = _sbss;
    *(.bss)
    *(.bss*)
    *(COMMON)

    . = ALIGN(4);
    _ebss = .;         /* define a global symbol at bss end */
    __bss_end__ = _ebss;
  } >RAM

  /* User_heap_stack section, check that there is enough RAM left */
  ._user_heap_stack :
  {
    . = ALIGN(8);
    PROVIDE ( end = . );
    PROVIDE ( _end = . );
    . = . + _Min_Heap_Size;
    . = . + _Min_Stack_Size;
    . = ALIGN(8);
  } >RAM
  
  /* Remove information from the compiler libraries */
  /DISCARD/ :
  {
    libc.a ( * )
    libm.a ( * )
    libgcc.a ( * )
  }
  .ARM.attributes 0 : { *(.ARM.attributes) }
}
\end{lstlisting}
这个文件中绝大部分不能更改,但是针对不同型号的单片机,有以下几个部分可以改动:
\begin{itemize}
	\item 第13行的LENGTH值,需要与单片机RAM大小相一致
	\item 第14行的LENGTH值,需要与单片机的FLASH大小相一致
	\item 第5行的\_estack值,需要等于第13行ORIGIN和LENGTH相加的结果
\end{itemize}

\subsection{makefile}
这个文件是构建工程的脚本,其模板如下:
\begin{lstlisting}[language=bash, style=customStyleMakefile, caption=makefile]
SRCS = main.c stm32f10x_it.c startup_stm32f10x_md.s 
HEADERS = stm32f10x_it.h stm32f10x_conf.x

########################## USER FILES BELOW ##########################
SRCS += 
HEADERS += 
########################## USER FILES ABOVE ##########################

PROJ_NAME = main

vpath %.c User
vpath %.a Lib
vpath %.s Startup 
vpath %.h User

CC=arm-none-eabi-gcc
OBJCOPY=arm-none-eabi-objcopy

CFLAGS  = -g -O2 -Wall -Tstm32_flash.ld 
CFLAGS += -mlittle-endian -mthumb -mcpu=cortex-m3 -mthumb-interwork
CFLAGS += -DUSE_STDPERIPH_DRIVER
CFLAGS += -DSTM32F10X_MD

CFLAGS += -ILib/inc -ICMSIS -IUser -lc

OBJS = $(SRCS:.c=.o)

.PHONY: lib proj

all: lib proj

lib:
	$(MAKE) -C Lib 

proj: $(PROJ_NAME).elf

$(PROJ_NAME).elf: $(SRCS)
	$(CC) $(CFLAGS) $^ -o $@ -LLib -lstm32periLib
	$(OBJCOPY) -O ihex $(PROJ_NAME).elf $(PROJ_NAME).hex
	$(OBJCOPY) -O binary $(PROJ_NAME).elf $(PROJ_NAME).bin

SRCS : HEADERS
HEADERS:

.PHONY: clean

clean:
	rm -f *.o *.elf *.hex *.bin
\end{lstlisting}
\par 
在编写代码的同时,这个文件中也有许多地方需要同时进行更改:
\begin{itemize}
	\item 每添加一个C源文件,需要在第5行正确地加上这个文件的文件名(不包含路径,但包含后缀名),以空格隔开,必要时可以使用反斜线“$\backslash$”作为续行符
	\item 每添加一个C头文件,需要在第6行正确地加上这个文件的文件名
	\item 每创建一个将会包含C源文件的文件夹,需要在第11行正确地加上这个目录的相对路径
	\item 每创建一个将会包含C头文件的文件夹,需要在第14行正确地加上这个目录的相对路径,并在第24行加上“-I”+目录的相对路径
	\item 当更换单片机的型号时,需要在第22行做出相应的调整,例如,对于低密度产品,要将“MD”替换为“LD”,对于高密度产品则需要替换为“HD”
\end{itemize}
上面的5点注意事项尤为重要,需要牢记在心。不过,由于这些改动较为机械,本文档在接下来的叙述中将不再做出强调,而默认读者会做出正确地操作。
\par 
完成上面的步骤后,我们的工程模板就创建好了,以后编写一个项目时,可以直接拷贝一份工程模板作为起始。当然,不建议直接在工程模板中编写代码。

\section{工程模板的编译与程序下载}
我们的工程模板实际上是一个完备的程序,无需做出任何更改就可以编译并下载到单片机上,这一过程可以作为我们平台搭建与工程模板创建正确性的检验环节。
\subsection{编译项目}
首先我们需要打开终端并将当前路径切换到工程模板的根目录下,如果使用VS Code编辑器,可以打开工程模板文件夹后直接调出集成终端,无需再切换路径。在终端中,输入“make”命令,按下回车。如果没有提示任何错误信息,并且工程模板中多出来了“main.bin”等文件,说明编译过程已经成功完成了。这里得到的“main.bin”就是我们需要的二进制程序。首次执行“make”命令可能会花费较长时间,不过执行过一次之后,此后的编译过程就会变得很快。这是因为make只会对上一次编译以来发生过改动的文件重新编译。
\subsection{下载程序}
在macOS上,下载程序需要用到st-link软件中的st-flash程序,同样打开终端并切换到工程模板根目录下,输入下面的命令:
\par 
\begin{lstlisting}[language=bash, style=customStyleBashLight]
st-flash write main.bin 0x8000000
\end{lstlisting}
\label{bash:download}
\par
如果已经将STM32通过ST Link连接上了计算机,并正确供电,那么此时程序就会写入单片机的Flash内部,只要执行上述命令时没有报错,按下单片机的复位键就可以运行程序了。当然,对于工程模板中的程序来说,并不会出现任何现象。
\subsection{清理项目}
由于执行make指令会产生一些没有作用的文件,如果我们想要清理这些文件,可以执行“make clean”指令。不过,这样做以后,下次编译就需要重新编译全部文件了。

\section{思考与练习}
\begin{enumerate}
	\item 在你的计算机上搭建开发环境,创建工程模板,并完成首次编译以及程序下载。
	\item 熟读并背诵更改makefile的注意事项。
	\item 考察\ref{bash:download}中的命令,其中0x8000000是指程序的起始地址,结合图\ref{memoryMap}说明为什么是这个值。
\end{enumerate}




















