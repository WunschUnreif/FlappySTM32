\renewenvironment{longtable}{\rowcolors{2}{LightGray}{white}\oldlongtable} {\endoldlongtable}
\chapter{开发环境搭建}
\section{开发环境概述}
编写STM32程序的过程大致可以分为3个阶段:编写代码、编译链接、下载程序。其中,编写代码是使用C语言,结合标准外设库,编制源代码的阶段,编译链接是利用编译器,将源代码编译成STM32平台的可执行文件,而下载程序是利用ST提供的接口驱动,将电脑上的可执行文件传输至STM32的Flash程序存储器中的过程。
下面针对这3个阶段,分别介绍本文档推荐的工具。
\paragraph{编写代码} 
本文档推荐的代码编辑工具是微软公司推出的开源跨平台编辑器VS Code。这款编辑器拥有丰富的插件,配合这些插件可以实现对C语言很好的支持,代码补全的能力相当高,而且可以一边编写代码一边查错。此外,利用它的集成终端功能,可以在编辑器内执行命令行命令,完成编译和下载。
\paragraph{编译链接}
本文档推荐的编译器是arm-none-eabi-gcc, 是由GCC提供的专为ARM平台编译程序的交叉编译工具链,历史悠久,性能可靠,最重要的一点是这个编译器完全免费而且开源,使用它不必担心遭到起诉,是彻头彻尾的正版软件。相比之下,Keil、IAR等软件虽然也是STM32程序编写和编译的流行软件,但它们正版软件的价格高昂,而且打击盗版的力度较大,因此不推荐使用。
\paragraph{下载程序}
本文档推荐的下载程序的接口和软件为ST-Link,这是ST公司为STM32设计的下载与调试接口,只需要4根线就可以连接。此外,使用J-Link也可以完成程序下载的功能,但J-Link接口需要20线,且下载器价格较高,因此并不推荐使用。目前,ST-Link也是ST公司官方推荐的连接方式。

\section{软件的安装}
VS Code的安装方法较为简单,只需在官网上下载安装,之后安装几个与C/C++有关的插件即可。
\par 
arm-none-eabi-gcc可以在https://github.com/gnu-mcu-eclipse/arm-none-eabi-gcc中找到下载链接,可以下载对应操作系统编译好的版本,也可以下载源码后自行编译。如果下载的是编译好的版本,那么在解压后得到的目录下的bin文件夹中就会看到arm-none-eabi-gcc这个可执行文件。方便起见,可以将这个目录添加到系统的环境变量PATH中。对于macOS用户(部分Linux系统用户也适用),可以在终端执行
\par 
\begin{lstlisting}[language=bash, style=customStyleBashLight]
nano ~/.bash_profile
\end{lstlisting}
\par 
然后在文件末尾添加:
\par 
\begin{lstlisting}[language=bash, style=customStyleBashLight]
export PATH="/path/to/gcc/bin:$PATH"
\end{lstlisting}
\par 
保存退出后,再执行命令:
\par 
\begin{lstlisting}[language=bash, style=customStyleBashLight]
source ~/.bash_profile
\end{lstlisting}
\par
即可完成环境变量的改变。
\par 
st-link软件可以直接利用系统的包管理器安装,对于使用homebrew的macOS用户,只需在终端中执行
\par 
\begin{lstlisting}[language=bash, style=customStyleBashLight]
brew install st-link
\end{lstlisting}
\par
稍等片刻就完成了st-link软件的安装。

\paragraph{Windows系统下相关软件安装概述}
对于Windows系统,代码编辑器VS code和交叉编译器arm-none-eabi-gcc都有相应的版本,只需下载安装即可,而对于下载软件st-link,也可用Windows系统下功能相同的软件替代。当然,我们推荐使用正版软件。

\section{建立工程模版}
首先,假定我们已经下载了STM32F1系列的标准外设库,并且位于STDPeriphLib/目录下。接下来,我们需要新建一个目录,作为工程模板的根目录,不妨设这个目录为C8Template/。在C8Template/下,我们需要新建下列文件夹和文件:
\begin{center}
	\begin{longtable}[l]{| p{30mm} | p{30mm} | p{80mm} |}
		\caption{C8Template/下的内容}\\
		\hline 
		\rowcolor{Gray}
		\textbf{项目名称} & \textbf{类型} & \textbf{简要描述} \\
		\hline
		\endfirsthead
		
		\hline 
		\rowcolor{Gray}
		\textbf{项目名称} & \textbf{类型} & \textbf{简要描述} \\
		\hline
		\endhead
		
		CMSIS/ &  目录 & 与Cortex-M3内核有关的库文件所在目录 \\ 
		Startup/ & 目录 & 单片机启动代码所在目录 \\
		Lib/ & 目录 & 外设库所在目录 \\
		User/ & 目录 & 开发者编写的代码所在目录 \\
		makefile & 文本文件 & 构建工程的脚本 \\
		stm32\_flash.ld & 文本文件 & gcc需要的连接器脚本 \\
		\hline
	\end{longtable}
\end{center}
\par 


























