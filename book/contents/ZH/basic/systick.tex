\renewenvironment{longtable}{\rowcolors{2}{LightGray}{white}\oldlongtable} {\endoldlongtable}
\chapter{SysTick定时器}

在上一章中,我们使用了循环的方式来进行延时,这种方法虽然易于编程,但是精度并不高,循环次数的确定并没有一种有效而又准确的方法。因此,对于时间精度要求较高的场合,我们应当考虑利用硬件资源来完成功能,毕竟,STM32外部的晶体振荡器的频率是相当精确的。

\section{SysTick概览}
Systick又叫系统滴答定时器,是包含在Cortex M3内核当中的一个核心外设。也就是说,这个器件与具体的芯片种类是无关的,只要使用了Cortex M3内核,就一定有这个外设。当然,也正因为SysTick要满足这种普适性,它并没有太多的功能,只是一个24位递减计时器。这意味着SysTick以一定的频率进行递减计数,当数到0时,它可以产生一个中断,同时自动重新填入计数初值,开始下一轮计数。当然,这个初值最大只能是$2^{24}-1$。
\par 
利用SysTick计数的特性,我们可以方便地实现延时、获取芯片运行时间等。在实时操作系统中,它还可以用来产生一个信号,让操作系统可以执行进程切换,实际上,这才是SysTick的最主要作用。
\par 
不过,在飞行控制中,我们通常不需要编写操作系统,所以下面我们还是利用SysTick来完成最基本的计时操作,重新实现上一章的延时函数,并完成一些额外功能。请注意,如无特殊说明,本部分的每一章都将沿用前几章的工作,包括引脚功能的约定、编写的程序文件等,也就是说,在这里我们沿用上一章中GPIO的相关函数。

\section{SysTick使用}
	\subsection{预备工作}
	这一部分的程序将写在User/文件夹下的SysTick.c和SysTick.h两个文件中。首先看一看头文件SysTick.h中的内容:
	\par 
	\begin{lstlisting}[language=bash, style=customStyleC, caption=SysTick.h]
#ifndef __SYSTICK__H__
#define __SYSTICK__H__

#include "stm32f10x.h" 
#include "stm32f10x_conf.h" 

void SysTickInit();
void DelayUsingSysTick(uint32_t ms);
void OnSysTick();

extern volatile uint32_t millis;

#endif
	\end{lstlisting}
	\par 
	可以看到,文件中声明了三个函数、一个变量,其中,SysTickInit()负责初始化SysTick定时器,使其每1ms产生一次中断;DelayUsingSysTick(uint32\_t ms)使用SysTick定时器进行延时,单位为ms;OnSysTick()为SysTick中断发生时的处理函数,我们将在SysTick的中断服务程序中调用这个函数。变量millis负责记录系统运行至今经过的毫秒数,它需要在SysTick.c文件中进行定义。下面,我们首先了解一下SysTick的初始化。
	
	\subsection{SysTick初始化}
	SysTick的初始化函数实现如下:
	\par 
	\begin{lstlisting}[language=bash, style=customStyleC, caption=SysTickInit函数]
void SysTickInit() {
	if(SysTick_Config(SystemCoreClock / 1000)) {
		while(1);
	}
}
	\end{lstlisting}
	\par 
	可以看到,其初始化工作非常简单,只需调用SysTick\_Config()这个函数即可。其参数为SysTick计数器的初值,意义是经过这么多次时钟脉冲之后会触发中断。此外,这个函数的调用会配置SysTick中断的优先级为最低的可能值,即15(或0xF),其原因与操作系统的调度有一定关系,在这里不过多解释。默认情况下,SysTick的时钟脉冲频率为72MHz,这也是SystemCoreClock这个宏展开后的值(即72000000),所以,配置计数器初值为SystemCoreClock / 1000即可使SysTick以1ms为周期触发中断。当然,在1ms内STM32大概能够执行$10^4$条指令,因此几乎无需担心SysTick中断不能在这段时间内处理完成。如果配置成功,调用SysTick\_Config()会返回0,否则返回非零值,因此我们需要判断这个返回值是多少,并在初始化失败的情况下让程序陷入死循环。这是因为SysTick是一个非常基本的器件,初始化失败往往意味着许多功能无法完成,甚至可能暗示芯片存在其他重大问题。
	
	\subsection{SysTick中断处理}
	由于SysTick中断服务函数位于stm32f10x\_it.c中,而我们希望将控制流转入OnSysTick()函数当中,为此,需要在stm32f10x\_it.c中包含头文件SysTick.h,然后在SysTick\_Handler中调用OnSysTick()函数:
	\par 
	\begin{lstlisting}[language=bash, style=customStyleC, caption=SysTick\_Handler函数]
void SysTick_Handler(void)
{
	OnSysTick();
}
	\end{lstlisting}
	\par 
	由于SysTick是一个Cortex M3提供的外设,其中断也属于内核,因此在stm32f10x\_it.c中,这个函数的定义可能已经存在,只不过默认情况下其中并没有任何语句。
	\par 
	将中断的控制流转向OnSysTick()函数后,就需要考虑其中该实现怎样的功能。我们希望,每次中断时,能够更新当前经过的毫秒数(增加1),并更新用于延时的计数器(减少1)。所以首先,我们需要在SysTick.c中定义变量millis用于计时,并且定义一个文件私有(可以用static)的变量用于延时:
	\par 
	\begin{lstlisting}[language=bash, style=customStyleC, caption=用到的全局变量]
volatile uint32_t millis = 0;
static volatile uint32_t timerForDelay = 0;
	\end{lstlisting}
	\par 
	需要注意volatile的使用,因为这两个变量是在中断时更新的,因此不能让编译器做出任何优化。接下来,只需要在OnSysTick()函数中更新它们的值即可。
	\par 
	\begin{lstlisting}[language=bash, style=customStyleC, caption=OnSysTick函数]
void OnSysTick() {
	++millis;
	if(timerForDelay) timerForDelay--;
}
	\end{lstlisting}
	\par 
	要注意的是,只有在延时计数器非零时,我们才去更新它的值。利用这个变量,我们就可以写出延时函数了。
	
	\subsection{利用SysTick延时}
	\par 
	\begin{lstlisting}[language=bash, style=customStyleC, caption=延时函数]
void DelayUsingSysTick(uint32_t ms) {
	timerForDelay = ms;
	while(timerForDelay);
}
	\end{lstlisting}
	\par 
	函数的操作很简单,设置延时计数器的值,然后等待它递减到0即可。
	\par 
	这个函数虽然用起来很方便,也更精确,但是,使用时有一些需要小心的地方。由于函数利用了SysTick中断,而如前文所述,初始化SysTick的时候其中断的优先级会被默认设置为最低,因此绝不能在其他中断处理程序中使用这个中断函数。当然,也正因此,如果SysTick中断发生时,单片机正在处理其他中断,那么SysTick中断的处理将被延后,这也会引来计时的误差。

\section{应用举例}
	下面我们用SysTick计时器的延时函数来重写上一章的部分例程。这些程序应写在main.c中,要注意包含SysTick.h头文件。
	\subsection{LED灯闪烁}
	完成LED灯闪烁的任务的主函数内容如下:
	\par 
	\begin{lstlisting}[language=bash, style=customStyleC, caption=LED闪烁]
int main(void)
{
	PinsInit();
	SysTickInit();
	
	while(1) {
		LED1Operation(1);
		DelayUsingSysTick(500);
		LED1Operation(0);
		DelayUsingSysTick(500);
	}
}
	\end{lstlisting}
	\par 
	这段程序的内容并没有什么需要解释的,下载到单片机上,将观察到LED以1Hz的频率不断闪烁。当然,这个功能也可以利用millis变量来实现。
	\par 
	\begin{lstlisting}[language=bash, style=customStyleC, caption=用millis实现LED闪烁]
int main(void)
{
	PinsInit();
	SysTickInit();
	
	u8 on = 1;
	while(1) {
		LED1Operation(on);
		while(millis % 500 != 0);
		on = !on;
	}
}
	\end{lstlisting}
	\par 

\section{小结}
总的来说,SysTick是一个非常简单的外设,当然其功能也较为单一,但是,SysTick的中断信号对于操作系统实现任务调度来说是至关重要的。我们知道STM32的内部只有一个CPU,因此只能同时运行一个进程,但是,我们仍然可以在其上编写多任务操作系统,原理就是以较高的频率在不同任务之间来回切换,造成多线程的表象。例如,时间片轮换是完成这种任务调度的最简单的方法,而它的核心代码其实只有300行左右。因此,操作系统并不一定是人们想象中那么复杂。随着你的水平不断提高,也许你也可以自己动手写操作系统。

\section{思考与练习}
\begin{enumerate}
	\item SysTick定时器的计数初值最大为多少?
	\item 如果SysTick的时钟频率为72MHz,那么可以配置的最长中断周期为多少?
	\item 是否可以在任何地方运行依赖SysTick的延时函数?有什么限制?
	\item 对于SysTick的中断处理函数,有什么运行时间的要求?
	\item 为SysTick模块添加全局变量seconds、minutes、hours,分别用于记录秒数、分钟和小时数,并在OnSysTick函数中正确地更新它们的值。指出它们加上millis这4个变量分别经过多久之后会发生溢出?设计程序使millis溢出时不影响seconds、minutes、hours的正确更新。
	\item 利用上一题实现的变量,制作一个数字时钟,可以用数码管作为显示设备。观察它的走时是否准确。尝试设计按钮,用于调整时间。
\end{enumerate}

















