\renewenvironment{longtable}{\rowcolors{2}{LightGray}{white}\oldlongtable} {\endoldlongtable}
\chapter{SysTick定时器}

在上一章中,我们使用了循环的方式来进行延时,这种方法虽然易于编程,但是精度并不高,循环次数的确定并没有一种有效而又准确的方法。因此,对于时间精度要求较高的场合,我们应当考虑利用硬件资源来完成功能,毕竟,STM32外部的晶体振荡器的频率是相当精确的。

\section{SysTick概览}
Systick又叫系统滴答定时器,是包含在Cortex M3内核当中的一个核心外设。也就是说,这个器件与具体的芯片种类是无关的,只要使用了Cortex M3内核,就一定有这个外设。当然,也正因为SysTick要满足这种普适性,它并没有太多的功能,只是一个24位递减计时器。这意味着SysTick以一定的频率进行递减计数,当数到0时,它可以产生一个中断,同时自动重新填入计数初值,开始下一轮计数。当然,这个初值最大只能是$2^{24}-1$。
\par 
利用SysTick计数的特性,我们可以方便地实现延时、获取芯片运行时间等。在实时操作系统中,它还可以用来产生一个信号,让操作系统可以执行进程切换,实际上,这才是SysTick的最主要作用。
\par 
不过,在飞行控制中,我们通常不需要编写操作系统,所以下面我们还是利用SysTick来完成最基本的计时操作,重新实现上一章的延时函数,并完成一些额外功能。请注意,如无特殊说明,本部分的每一章都将沿用前几章的工作,包括引脚功能的约定、编写的程序文件等,也就是说,在这里我们沿用上一章中GPIO的相关函数。

\section{SysTick使用}
	\subsection{预备工作}
	这一部分的程序将写在User/文件夹下的SysTick.c和SysTick.h两个文件中。首先看一看头文件SysTick.h中的内容:
	\par 
	\begin{lstlisting}[language=bash, style=customStyleC, caption=SysTick.h]
#ifndef __SYSTICK__H__
#define __SYSTICK__H__

#include "stm32f10x.h" 
#include "stm32f10x_conf.h" 

void SysTickInit();
void DelayUsingSysTick(uint32_t ms);
void OnSysTick();

extern volatile uint32_t millis;

#endif
	\end{lstlisting}
	\par 
	可以看到,文件中声明了三个函数、一个变量,其中,SysTickInit()负责初始化SysTick定时器,使其每1ms产生一次中断;DelayUsingSysTick(uint32\_t ms)使用SysTick定时器进行延时,单位为ms;OnSysTick()为SysTick中断发生时的处理函数,我们将在SysTick的中断服务程序中调用这个函数。变量millis负责记录系统运行至今经过的毫秒数,它需要在SysTick.c文件中进行定义。下面,我们首先了解一下SysTick的初始化。
	
	\subsection{SysTick初始化}
	SysTick的初始化函数实现如下:
	\par 
	\begin{lstlisting}[language=bash, style=customStyleC, caption=SysTick.h]
void SysTickInit() {
	if(SysTick_Config(SystemCoreClock / 1000)) {
		while(1);
	}
}
	\end{lstlisting}
	\par 
	可以看到,其初始化工作非常简单,只需调用SysTick\_Config()这个函数即可。其参数为SysTick计数器的初值,意义是经过这么多次时钟脉冲之后会触发中断。此外,这个函数的调用会配置SysTick中断的优先级为最低的可能值,及15(或0xF),其原因与操作系统的调度有一定关系,在这里不过多解释。默认情况下,SysTick的时钟脉冲频率为72MHz,这也是SystemCoreClock这个宏展开后的值(即72000000),所以,配置计数器初值为SystemCoreClock / 1000即可使SysTick以1ms为周期触发中断。当然,在1ms内STM32大概能够执行$10^4$条指令,因此几乎无需担心SysTick中断不能在这段时间内处理完成。如果配置成功,调用SysTick\_Config()会返回0,否则返回非零值,因此我们需要判断这个返回值是多少,并在初始化失败的情况下让程序陷入死循环。这是因为SysTick是一个非常基本的器件,初始化失败往往意味着许多功能无法完成,甚至可能暗示芯片存在其他重大问题。












