\chapter{绪论}

\section{为什么要学习STM32}
自动化控制技术在飞行器制造的过程中日趋重要,姿态稳定、舵面控制等都要用到电子控制,而单片机以其重量轻、耗电少、控制逻辑灵活、易于学习应用等诸多优势,成为航模电子系统的首选控制器类型。在各类单片机中,STM32具有很高的性价比:丰富的片上外设、完备的库函数、灵活的控制方式,以及远高出同类产品的时钟频率,使其成为在各个领域都有很强通用性的单片机产品,而它同样适合作为飞行控制器的核心,因此,我们有必要学习这款单片机,了解它的编程方式、具体功能,以便开发更为强大的控制系统。
\par 
(以上都是胡编的,我也不知道为什么要学习STM32。。。)

\section{如何学习STM32}

作为一种应用技术,STM32的最佳学习方式自然是实践,因此希望读者在阅读本文档的同时,能够多动手进行编程实践。虽然本文档的定位是一份涵盖作者认为编程中所有可能遇到的问题的事无巨细的指南,但不可否认仍有许多的问题只有亲自动手才能发现并解决,从而作为日后的经验积累下来。在这个过程中,我们也欢迎读者对本文档的内容做出斧正与完善。

\par 
此外,官方文档始终是学习STM32的权威帮助,作者在此推荐STM32的Datasheet和Reference Manual,以及其标准外设库的文档,其中包含大量的示例代码以供学习参考。当然,遇到难以解决的问题时,求助网络可能是最快的解决方式,毕竟STM32在国内拥有相当大的群众基础,大多数问题都有网友进行总结了。

\section{预备知识}

本文档认为读者已经基本掌握了下列预备知识:
\begin{itemize}
	\item 计算机基本操作及其大致原理的了解
	\item C语言程序设计(本文将采用C11标准),包括熟练掌握2阶以下的指针、函数指针、结构体、枚举等知识
	\item 电子技术,只需掌握简单的电路分析、一些数字电路的基本知识,但要求有一定的实践基础,例如能够正确地连接电路
\end{itemize}

\section{局限性}
本文档编写的目的是阐述STM32单片机在一般的飞行控制中需要用到的外设的编程控制方法,对于其他的外设讨论很少。此外,本文档将着重介绍STM32F1系列的单片机,而并不会涉及性能更强的STM32F4、STM32F7等系列的单片机。

\section*{}
本章的最后,祝愿各位读者在今年上半年迅速掌握STM32这款单片机,软硬两开花!
